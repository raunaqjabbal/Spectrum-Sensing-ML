\chapter{Discussions and conclusion}
\section{Contributions}
In this report, we have thoroughly discussed effects of various parameters like fading channels, number of SUs, sensing time, training dataset size for Spectrum Sensing by applying a wide variety of algorithms of various types, using metrics like ROC curve and AUC value to compare performance.
Results show that Machine Learning algorithms outperform both Classical Cooperative algorithms and Non-cooperative algorithms, with Linear SVM, Gaussian SVM, Logistic Regression and MLP showing the best performance, and Machine Learning algorithms have a lot of potential in this field, as we can get acceptable performance after learning from only 50 samples. The number of SUs dictate the performance, especially when the number of SUs are low. AWGN scenario is most favourable, followed by Rician Fading and Nakagami Fading (order depends on the M parameter value), and then Rayleigh Fading, but we can increase performance by changing other parameters like having more SUs, sensing for longer intervals. 


\section{Limitations}
Other Deep Learning Models like Convolutional Neutral Netowrks (CNNs) have not been used. SUs are not randomly distributed, as the performace of various algorithms and fading scenarios have to be compared. Effect of SNR on the Spectrum sensing performance has not been shown.


\section{Future scope}
Interesting aspects that should be considered in the future, is using Deep Reinforcement Learning to have a CRN that learns on itself requiring little human interference. Mobile CRNs are CRNS where either the SUs are mobile or the PU is mobile. Mobile CRNS and the tuning of the parameters to bring the performance close to the performance of static CRNs have a lot of scope as a lot of  consumer devices are mobile.