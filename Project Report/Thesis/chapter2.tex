\chapter{Literature review}
\label{mathchapter}


\section{Preamble}
This chapter provides background to the Spectrum Sensing problem and the research done by various authors along with the gaps in research in this field of study. This chapter also analyzes the problem and formulates it.


\section{Background}
The requirement for higher bandwidth keeps increasing due to innovation in wireless technologies, and it is apparent that we need an intelligent and dynamic system to resolve this issue. Various campaigns have found that static spectrum access leads to overcrowding in some parts of the spectrum and underutilization in others [16]. This imbalance reduces the effective utilization of the spectrum. We require Cognitive Radio, which enables wireless devices to transmit in spectrum holes as long as the PUs are not transmitting. Cognitive Radios main objective is to use the best spectrum that the PU is not using. Cognitive Radio networks sense opportunities, characterize the environment, determine the best strategy for decisions, and adapt by changing operation parameters. The environment is dynamic, and Cognitive Radio should keep track of the changes in the environment. We mainly focus on the Spectrum Sensing part, the most crucial part of Cognitive Radio Networks. Different techniques for Spectrum Sensing are:
\begin{itemize}	
\item Energy Detection
\item Cyclostationary detection
\item Matched Filter Detection
\end{itemize}
Energy Detection has been widely used as it is the most straightforward technique, it needs less sensing time, and we need no prior knowledge about the PU or its signal. The downside of Energy Detection is the poor performance we obtain when the Signal to Noise Ratio (SNR Value) is low. Cyclostationary detection exploits the second-order periodicity of the modulated signal and provides good results even at low SNR values but is complex to compute. The matched Filter technique maximizes the SNR value of the detected signal, but prior knowledge of the signal is required.


\section{Key related research}
In [16], it is observed that the average occupancy of the spectrum is very low, giving potential to the idea of cognitive Radio and Spectrum Sensing, to maximise the usage of the spectrum band. In [1,5,15,21] compare Machine Learning algorithms to classical algorithms (AND, OR, MRC). Authors have implemented K Means Clustering in [1], and Naïve Bayes, SVM, MLP in [5,21]. Authors in [15] implement various Machine Learning Algorithms, and provide various metrics like Accuracy, Precision, Recall. SVM being a robust algorithm, is used in various papers [5,8,11,12,14,15,17,19,21,23,24,26,29] using various kernels like linear, gaussian, polynomial. Naïve Bayes algorithm, has also been implemented in [5,8,15,21,27]. 

Deep Learning algorithms have been very popular to solve various issues related to Cognitive Radio. In [28], authors have used a type of Deep Learning model called a Convolutional Neural Network (CNN), mainly used for image recognition and classification. They are good at recognizing complex patterns in the given input (like gradients, shapes, and lines). They have focused on various modulation types (64QAM, B-FM, BPSK, CPFSK, DSB-AM, GFSK, PAM4, QPSK, SSB-AM), and the CNN models try to classify the input data and link it to one of these modulation types. They have also compared the performance of models in different published papers related to modulation types. Authors in [25] mainly emphasize on Context Awareness in Wireless Communications that will improve the efficiency of existing services for which the authors have used Machine Learning and Deep Learning algorithms in various categories like Unsupervised Learning, Supervised Learning, and Reinforcement Learning. The authors have not focused on spectrum sensing. In [3], Deep Reinforcement Learning technique is used, which does not require a labeled dataset and can adapt to the environment with little human interaction and can work dynamically in the environment. It works by rewarding or punishing the model for making decisions. Authors in [19] like [3], use Deep Reinforcement Learning technique to decide whether PU uses the spectrum band. Multiple PUs and a wideband channel are considered. K out of N, SVM, and DL models also have been compared for performance. In [17], Artificial Neural Network, SVM, Decision Tree, and KNN were used to detect PUs presence. The dataset was generated using an Arduino Uno Card and a wireless transmitter, which is a practical dataset. Authors in [22] used: Neural Networks, Expectation-Maximization, and K-Means Clustering. The authors tested models on simulated and real signals to show theoretical and practical performance. Authors in [24] consider SVM, CNN, and Deep Reinforcement Learning models for Cognitive Radio and discuss how several aspects need improvement when applying algorithms for Cooperative Spectrum Sensing.

A few papers have gone in depth to solve a particular issue or considered various other parameters. In [18], authors did not use any Machine Learning algorithms, they have concentrated on SNR Walls, a threshold below which, no matter how long a detector senses, will fail to be robust because it gets tough to distinguish between the h0 and h1 hypotheses. They also discuss what happens on the other side of the SNR Wall, its impact, Spectrum Holes and SNR Walls in space, and how metrics reveal trade-offs and the importance of diversity. In [29] The paper considers that the PU transmits with discrete power levels with set probabilities, and K Means Clustering is used first to label the dataset, and SVM learns from the dataset and the labels to predict if the PU is transmitting or not. In [8] SUs operate under a Hybrid Underlay-Interweave Model, meaning that the SUs can utilize the spectrum when the PU is not transmitting and simultaneously access it along with the PU while abiding by the Interference Temperature. Multiple SVMs, Gaussian Mixture Model (GMM), and Naive Bayes algorithms have been used to decide if the PU is active. Authors in [7] focus on showing the effects of Malicious Users (MUs) and mitigating their effects. MUs send false data to the Fusion Centre, which would affect the performance of Cooperative Spectrum Sensing. The authors have proposed a Hybrid Boosted Tree Algorithm based on Differential Evolution and the Boosted Tree Algorithm. This model has been compared to other models like the Genetic Algorithm and KNN. The authors have provided detailed results by running simulations with varying parameters like SNR values and population sizes.

Various papers [4,9,25,28] review, condense models and scenarios from other papers. Authors in [9] have published a survey including various types of algorithms under Supervised Learning, Unsupervised Learning, and Reinforcement Learning algorithms from different branches of Machine Learning while authors in [4] focus on Conventional techniques and Advanced techniques, including various methods like Covariance-Based Sensing and Machine Learning techniques.


\section{Analysis}
For Cognitive Radio to work as intended, the algorithms used should come to the correct decision and should come to that decision without taking much time. If the Spectrum Sensing decision is incorrect, the SU may fail to sense a spectrum hole or utilize the spectrum when the PU is still actively transmitting, when an SU should not interfere with the PU. If Spectrum Sensing takes too long, we are wasting time that the SUs could have utilized to occupy the spectrum band.

The downside of Non-cooperative Spectrum Sensing, as the name suggests, is that the decision does not involve cooperation. The environment may lead the SU to consider the wrong decision. Another issue is to find a suitable threshold for the SU. Classical algorithms can detect more efficiently, but now we have to figure out threshold values for multiple SUs at various distances from the PU. An efficient implementation of the MRC algorithm also requires us to have an estimate of the SNR values. Machine Learning algorithms do not have the abovementioned issues.

Machine Learning models have been prevalent to improve the performance of Spectrum Sensing. Various papers have even tried to incorporate Deep Learning models as Deep Learning is more promising [4] [7] [13] [18] [25] [26]. Along with Non-cooperative Spectrum Sensing and Classical algorithms, this paper aims to compare the performance of a wide variety of Machine Learning algorithms because one algorithm may consistently outperform other algorithms in a few scenarios, and other algorithms may outperform in other scenarios. Performance of Gradient Boosted Algorithms has also been depicted. This paper also compares how these algorithms perform under fading scenarios like Rayleigh Fading, Rician Fading, and Nakagami Fading.


\section{Problem formulation}
The primary goal of Spectrum Sensing is to differentiate between $H_{0}$ and $H_{1}$ (Fig. 1.1) with high accuracy.

Performance of various algorithms, fading scenarios (h), sensing samples (t), SU numbers, training dataset sizes will be calculated, along with with the most optimal values to give a comperehensive review to show how each of these parameters affect Spectrum Sensing.

\section{Research gaps}
No paper gives a detailed and concise description of performance of Spectrum Sensing considering a wide variety of parameters like Sensing Time, SU numbers, Training Dataset Size and various cooperative Spectrum Sensing algorithms. Performance of Gradient Boosting algorithms like XGBoost, CatBoost, and ADABoost have not yet been depicted. 