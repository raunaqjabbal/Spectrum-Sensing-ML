\chapter{Data Generation Code}

\textbf{MCS:}

function [Y,A,PU,n,Z,SNR] = MCS(scenario)

M = 1;                    // Number of PUs

N = size(scenario.SU,1);                 // Number of SUs derived from counting number of locations of SUs

txPower = scenario.TXPower;       // Transmission power

noisePower = scenario.NoisePower*ones(N,1); // Gaussian noise power

a = 4;                                      // Path-loss exponent

samples = round(2*scenario.T*scenario.w);   // Number of samples

realiz = scenario.realiz;                   // Number of Monte-Carlo realizations

fading=scenario.fading; 		// Fading type
    
// Compute the Euclidean distance for each PU-SU pair

d = zeros(N,1);

    for i=1:N

        d(i) = norm(scenario.PU(:,:)-scenario.SU(i,:));

        if (d(i) == 0)

            d(i) = 1;

        end

    end

// Main
Y = zeros(realiz,N);                        // Power estimated

S = zeros(realiz,1);                          // Channel availability

PU = zeros(N,samples,realiz);               // PUs signal received at SU

n = zeros(N,samples,realiz);                // Noise received at SU

Z = zeros(N,samples,realiz);                // PU signal + noise received at SU

SNR = zeros(N,realiz);                      // PU SNR at the SU receiver

for k=1:realiz

    n(:,:,k) = gaussianNoise(samples,noisePower);       // Get the noise at SU receivers

    H = channel(N,d,a,fading,scenario.variance);                      // Get the channel matrix             

    [X, S(k)] = PUtx(samples,txPower, scenario.Pr); // Get the PU transmissions for 50// activity of PU

    for i=1:N

        for t=1:samples

            PU(i,t,k) = PU(i,t,k) + H(i)*X(t);

            Z(i,t,k) = PU(i,t,k) + n(i,t,k);

        end

        SNR(i,k) = mean(abs(PU(i,:,k)).*abs(PU(i,:,k))/noisePower(i);		// Calculating SNR

        Y(k,i) = sum(abs(Z(i,:,k)).*abs(Z(i,:,k)))/(noisePower(i)*samples); // Normalized by noise variance

    end

end

A=S(:,1); // Channel availability






\textbf{PUTx.m:}
function [X,S] = PUtx(samples,txPower, Pr)

// Get the PU transmission

// PUtx(M,samples,txPower, Pr)

// M - Number of PUs

// samples - Number of transmission samples

// txPower - Average transmission power for each PU

// Pr - Active probability for each PU

S = zeros(1); // PU states

X = zeros(samples,1); // Signal at PU transmitters

    p = rand(1);

    if (p <= Pr)

        S(:) = 1;

    end


if (S>0)

    X = normrnd(zeros(samples,1),ones(samples,1).*sqrt(txPower)); //Signal Data Generation

end




\textbf{gaussianNoise.m:}

// Returns the Gaussian Noise with zero mean and unitary variance

// gaussianNoise(N,samples,noisePower)

// N - Number of SUs

// samples - Number of noise samples 

// noisePower - Noise power for each SU

N = size(noisePower,1);

n = zeros(N,samples);

for i=1:N

    n(i,:) = normrnd(0,real(sqrt(noisePower(i))),1,samples);

end

end


\textbf{channel.m:}

function H = channel(N,d,a,type,variance)

// Calculate the channel matrix (path-loss + fading)

// H = channel(M,N,d,a)

// type: type of faing used

// variance: variance value used

// M: Number of PUs

// N: Number of SUs

// d: Euclidean distance between each PU-SU pair

// a: Path-loss exponent

H=zeros(N,1);

if (strcmp(type,'ray'))

    H=sqrt(-2 * variance * log(rand([N,1])))/sqrt(2);

elseif (strcmp(type,'nakagami'))

    m=1.5;

    omega=1;

    H = sqrt(gamrnd(m,omega/m,[N,1])).*sqrt(variance);

elseif (strcmp(type,'rician'))

    for i=1:N

        ricChan= comm.RicianChannel("MaximumDopplerShift",0,"NumSamples",1,"ChannelFiltering",0);

        H(i,:) = abs(ricChan());

    end

    // H=H.*sqrt(variance);

else

    H = ones(N,1);

end

H = H.*sqrt(d.\^(-a));// Fading + path-loss (amplitude loss)

end
